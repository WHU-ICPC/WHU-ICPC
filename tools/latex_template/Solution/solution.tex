\documentclass{beamer}
%\setbeamertemplate{navigation symbols}{}
\usepackage{xeCJK}
\usepackage[orientation=landscape,size=custom,width=16,height=9,scale=0.5,debug]{beamerposter}%修改比例,16:9
\usepackage{eqnarray}
%\usepackage{zhfontcfg_mac}
\usepackage{enumitem}
\usepackage{tikz,pgfplots,pgfplotstable}
%\usepackage{mathptmx}
\usepackage{eulervm}
\usepackage[T1]{fontenc}
\usepackage{beamerthemeshadow}
\usepackage{qrcode}
\usepackage{ulem}
\usefonttheme{professionalfonts}

\setlength{\parskip}{0.5\baselineskip}

\begin{document}
\title{2021 Wuhan University Freshman Programming Contest}
\author{WHU ICPC 集训队}
\date{Apr. 4th, 2022} 

\newcommand{\problemname}{A}

\begin{frame}
\titlepage

\end{frame}

\makeatletter
\newcommand\resetstackedplots{
\makeatletter
\pgfplots@stacked@isfirstplottrue
\makeatother
\addplot [forget plot,draw=none,x=P,y=z] coordinates{\problemdata};
}
\makeatother

\begin{frame}\frametitle{比赛小结}

\begin{itemize}
\item[$\cdot$] 本次比赛共收到 777 份提交代码。
\item[$\cdot$] 其中 185 份代码正确。
\item[$\cdot$] 82 名参赛选手有提交记录。
\item[$\cdot$] 73 名参赛选手至少通过一题。
\end{itemize}

\end{frame}

\begin{frame}\frametitle{各题通过情况}
    \pgfplotstableread{
	P AC NP
	A 58 118
	B 0 2
	C 1 10
	D 17 143
	E 1 5
	F 2 49
	G 16 83
	H 70 76
	I 0 5
	J 8 29
}\subdata
    \begin{tikzpicture}
	\pgfplotsset{                ybar stacked,
		y axis line style = { opacity = 0 },
		% axis x line       = none,
		width=\textwidth,
		height=.75\textheight,
		legend style={at={(0.5,1)},
			anchor=south,legend columns=-1},
		symbolic x coords={A,B,C,D,E,F,G,H,I,J},
		xtick=data,
		ymajorgrids=true,
		yminorgrids=true,
		xmajorgrids=false,
		xminorgrids=false,
		xminorticks=false,
		minor y tick num=3,
		xtick style={draw=none},
		ymin=0,ymax=180,
		ylabel={\# of submissions},
		grid style=dashed,}
	
		\begin{axis}[bar shift=3pt,bar width=6pt,legend style = {at={(0.3,1)},anchor=south west}]
		\addplot [fill={rgb,255:red,156; green,255; blue,87}] table [x=P,y=AC]{\subdata};
		\addplot [fill={rgb,255:red,255; green,0; blue,0}] table [x=P,y=NP]{\subdata};
		\legend{通过,未通过}
		\end{axis}
\end{tikzpicture}
\end{frame}

\renewcommand{\problemname}{A. 仓鼠快速签到}

\begin{frame}\frametitle{\problemname}
	\begin{block}{解法}
		前 7 道题的限制唯一确定了全部 10 道题的答案,因此选手完全不需要知道后 3 题的答案。

		对于完全不会写代码的萌新,这题可以手撕。可以从限制性比较强的题目入手,用常识也可以排除后 3 题的部分选项,请不要放弃。

		对于会写 for 和 if 的选手,枚举每道题的答案共 $3^{10}$ 种情况,再对前 7 题的限制进行验证即可。代码可以写的很短。

		如果你不会写第 7 题的限制,前 6 道题限制后也只有 5 种情况,再手动验证第 7 题即可。
	\end{block}
\end{frame}

\renewcommand{\problemname}{B. 二维弹球}

\begin{frame}\frametitle{\problemname}
    \begin{block}{题意}
		\textbf{计算几何入门模拟题}。

		碰撞时,\textbf{沿边的方向速度不变,垂直于边方向的速度反向}。

		给定初始位置、速度向量、时间,求最终位置。
    \end{block}

	\begin{block}{出题人的话}
	计算几何好难出,凸包生成了很多 $n=20$ 的包,只能手动构造 $n=100$ 的。希望大家善待计算几何。
	\end{block}
\end{frame}

\begin{frame}\frametitle{\problemname}
\begin{block}{样例2的图示}
	\begin{tikzpicture}[]


		\draw [->](-1.25,0) -- (2.25,0)node[above]{$x$};
		\draw [->](0,-2.5) -- (0,1.5)node[right]{$y$};
		
		\node at (-0.5,0)[below]{$O$};
		
		\draw [thick](2,0) -- (0,1) -- (-2,0) -- (-1,-2) -- (1,-2) -- (2,0); 
		
		\draw[->, red, thick] (0.000,0.000) -- (0.667,0.667) -- (0.286,-2.000) --(-0.133,0.933) -- (1.600,-0.800) -- (-1.846,-0.308) -- (-1.556,0.222) -- (-0.560,0.080);
	\end{tikzpicture}
\end{block}
\end{frame}

\begin{frame}\frametitle{\problemname}
	\begin{block}{题解}
		根据题目描述,进行模拟。

		注意到 $t\le 100$,可按时间模拟。但可能一秒钟出现多次碰撞,判断起来较为麻烦。
		
		注意到 碰撞次数 $\le 10^4$,按碰撞模拟。
		
		常用的“点+向量=点”技巧,球当前位置是一个点,速度向量与时间做数乘是下一个位置。可以用当前位置与下一个位置的连线所在的直线判断下一个碰撞点,这时需要用到的知识点是\textbf{直线交点}。
	\end{block}
\end{frame}


\begin{frame}\frametitle{\problemname}
	\begin{block}{题解}
碰撞时,\textbf{沿边的方向速度不变,垂直于边方向的速度反向}。

那么这时需要旋转速度向量。那么旋转多少度呢?求出\textbf{速度向量与边向量}的夹角 $\theta$,逆时针旋转 $2\theta$ 即可。可以证明,无论 $\theta$ 是锐角还是钝角,都是正确的。

所以我们只需要求出每一次碰撞的位置,算出碰撞间隔时间,就可以找到圆最终的位置。
	\end{block}
\end{frame}

\begin{frame}\frametitle{\problemname}
	\begin{block}{锐角的情况}

\begin{tikzpicture}[]
		
	\node at (1.9,1.8)[below]{$\theta$};
	\draw[blue,line width=1] (2,1.8) arc (270:225:.2);
	
	\draw [->,thick](2,1) -- (2,3); 
	
	\draw[->, red, thick] (0,0) -- (2,2);
\end{tikzpicture}\qquad\qquad\qquad\qquad\begin{tikzpicture}[]

\node at (1.9,1.7)[below]{$\theta$};
\draw[blue,line width=1] (2,1.8) arc (270:225:.2);

\draw [->,thick](2,1) -- (2,3); 

\draw[->, red, thick] (0,0) -- (2,2) -- (0,4);

\draw[->, red, thick, dashed] (2,2) -- (4,4);
\draw[blue,line width=1] (2.166,2.166) arc (45:140:.2);
\node at (2.1,2.1)[above]{$2\theta$};
\end{tikzpicture}\end{block}
\end{frame}

\begin{frame}\frametitle{\problemname}
	\begin{block}{钝角的情况}
\begin{tikzpicture}[]
	
	\node at (1.8,2)[above]{$\theta$};
	\draw[blue,line width=1] (2.166,2.166) arc (45:180:.2);
	
	\draw [->,thick](3,2) -- (1,2); 
	
	\draw[->, red, thick] (0,0) -- (2,2);
	\draw[->, red, thick, dashed] (2,2) -- (4,4);
\end{tikzpicture}\qquad\qquad\begin{tikzpicture}[]
	
	\node at (1.7,2)[above]{$2\theta$};
	\draw[blue,line width=1] (2.166,2.166) arc (45:315:.2);
	
	\draw [->,thick](3,2) -- (1,2); 
	
	\draw[->, red, thick] (0,0) -- (2,2) -- (4,0);
	\draw[->, red, thick, dashed] (2,2) -- (4,4);
\end{tikzpicture}
\end{block}
\end{frame}
\begin{frame}\frametitle{\problemname}
	\begin{block}{题解}
		因此对一个向量 $(x,y)$,逆时针旋转 $\theta$,所得到的新向量为 $(x\cos\theta-y\sin\theta,y\cos\theta+x\sin\theta)$。

		求两线段交点比较容易,用面积法和向量去找。

		总的时间复杂度为 $O(n$碰撞次数$)$。
	\end{block}
\end{frame}

\renewcommand{\problemname}{C. osu! catch}

\begin{frame}\frametitle{\problemname}
    \begin{block}{题意}
	有$N$个水果,第$i$个水果会于$t_i$时刻末在位置$p_i$落下,如果玩家在此时恰好移动到同一位置,就接住了这个水果。

	有以下3种移动方式,分别是原地不动,移动一个位置,或等概率移动1个位置到k个位置。

	求在最优的移动策略下,接到水果个数的期望。
	\end{block}
	
\end{frame}

\begin{frame}\frametitle{\problemname}
	\begin{block}{解法}
		注意到要求解的是接住水果个数的\textbf{期望},因此我们需要倒序对水果进行dp。

		设$dp(i, j)$表示时刻$i$初,玩家处在位置$j$,后续能接住个数的最大期望值。

		列出3种转移方式,从时刻$i+1$向时刻$i$转移。
		
		\vspace{-0.2in}
		$$
		dp(i, j)=v(i,j)+\max\begin{cases}
			dp(i+1, j),\ dp(i+1, j-1),\ dp(i+1, j+1)&\\
			\frac{1}{k}\sum_{d=j-k}^{j-1}dp(i+1, d),\, \frac{1}{k}\sum_{d=j+1}^{j+k}dp(i+1, d)&\\
		\end{cases}
		$$

		其中,$v(i,j)$ 表示 i 时间 j 位置是否落下水果。

		区间和可以用前缀和优化,注意特判快速移动遇到边界的情况。

		最后答案为$\max_{1\leq j \leq m}dp(1, j)$

		这种做法的复杂度为$O(m*t_n)$。
	\end{block}

\end{frame}

\renewcommand{\problemname}{D. 和谐之树}

\begin{frame}\frametitle{\problemname}
	
	\begin{block}{题意}
		求对区间 $[1,n]$ 建立线段树后最大节点编号
	\end{block}
\end{frame}

\begin{frame}\frametitle{\problemname}
	
	\begin{block}{题解}
		考虑最大节点编号在当前左子树还是右子树。

		如果左边比右边深度更深,那么在左子树,否则在右子树。

		可以发现,当区间长度从 $2^x$ 变到 $2^x+1$ 长度时,线段树深度会 +1.

		快速比较两个子树深度即可
	\end{block}
\end{frame}
\renewcommand{\problemname}{E. 和谐之树·改}

\begin{frame}\frametitle{\problemname}

    \begin{block}{题意}
        区间询问 D 题的最大编号之和。
    \end{block}

\end{frame}

\begin{frame}\frametitle{\problemname}
	\begin{block}{题解}
        考虑选择左子树的时候,左边区间长度必须是 $2^x+1$,右边长度必须是 $2^x$,那么选择左子树时,当前区间的二进制必须形如 $100\dots001$.

        考虑从当前长度 $n$ 的节点走到某棵子树时,会变成 $n>>1|(1 \ or \ 0)$,那么走左子树的过程最多只会发生一次,即最开始长度的二进制表示中,第二高位 $1$ 的位置处。

        进一步发现,区间长度 $n$ 最高位的两个 $1$ 确定后,最大编号也确定了。那么 1e18 内的不同最大编号只有 $\log^2v$ 级别。打表同样也能发现。

        预处理后即可回答。注意 $T log^2 v$ 的 $log$ 比较大,可能不能通过。
    \end{block}
\end{frame}
\renewcommand{\problemname}{F. 仓鼠与炸弹}

\begin{frame}\frametitle{\problemname}
    \begin{block}{题意}
        问字符串 $S$ 有多少个字串 $T = S[l,r]$ 满足:

		\begin{itemize}
			\item 1. $T$ 的长度大于等于 $2m$
			\item 2. $T[1,m] = \mathrm{reverse}(T[|T|-m+1,|T|])$
		\end{itemize}
    \end{block}

    \pause

	\begin{block}{解法}
	字符串哈希:

	定义: 字符串 $S = s_1, s_2, \dots s_n$ 的哈希为 $Hash(S) = (s_1 a^1 + s_2 a^2 + \dots + s_n a^n) \pmod{p}$

	我们认为,如果两个字符串的 hash 值相等,那么他们\textbf{大概率}是相等的
	
	同时,我们可以 $O(1)$ 地求出 $Hash(S[l,r])$ 和 $Hash(\mathrm{reverse}(S[l,r]))$
	\end{block}
\end{frame}

\begin{frame}\frametitle{\problemname}
	\begin{block}{解法}
	从左到右枚举 $i$,找到有多少个位置 $j\le i-m$ 满足 $Hash(\mathrm{reverse}(S[i-m+1,i]))=Hash(S[j-m+1,j])$
	复杂度 $O(n)$ 或者 $O(n\log n)$
	\end{block}

	\begin{block}{细节}
	如果只对一个 $10^9$ 级别的质数取模的话,大概率会 WA(生日悖论)

	如果对 $2^{64}$ 次方自然溢出取模的话,可以构造出两个不同的长度在1000左右的字符串,他们的哈希值相同
	\end{block}
\end{frame}
\renewcommand{\problemname}{G. 寄寄子的生日}

\begin{frame}\frametitle{\problemname}

    \begin{block}{题意}
		给 $2\sim n$ 排列,只能交换值互质的两个数,构造一个 $2n$ 次交换后有序的方案。
    \end{block}

\end{frame}

\begin{frame}\frametitle{\problemname}
	
	\begin{block}{题解}
		考虑通过找到最大的质数作为中间点交换,可以在两步内将任意一个数归位。

		具体的 \texttt{a b p -> p b a -> b p a},假如 $p$ 是最大质数,那么可以把 $b$ 归位。

		关于 $2\sim n$ 内最大的质数 $p$,一定有 $2\times p>n$ 成立,至少在题目范围内是正确的。

		由于 $n$ 开的比较小,因此直接暴力也可以通过,主要是想多给大家一点签到题。
	\end{block}
\end{frame}		


\renewcommand{\problemname}{H. wwhhuu}

\begin{frame}\frametitle{\problemname}
	\begin{block}{题意}
		构造一个长为 $n$ 的字符串,使子序列 \texttt{whu} 个数最多。
	\end{block}
\end{frame}

\begin{frame}\frametitle{\problemname}
	\begin{block}{题解}
		题目名已经把做法写出来了,显然 www...hhh...uuu... 的相应子序列个数最多。

		注意 $w,h,u$ 谁多谁少并不重要,只要三者个数差小于等于 $1$ 即可。
		
		证明可以考虑调整法,如果不是 www...hhh...uuu... 往这个方向调整不会变差,如果个数差大于 $1$ 往小于 $1$ 调整答案也不会变得更差。
	\end{block}
\end{frame}
\renewcommand{\problemname}{I. 异度之刃}

\begin{frame}\frametitle{\problemname}

\begin{block}{题意}
    线段树套路题

    不带修改的区间本质不同连续上升子串
\end{block}

\begin{block}{题解}
    对于不带修改的题目,可以考虑将询问离线,移动右端点,每次移动后维护左端点的答案
\end{block}

\end{frame}

\begin{frame}\frametitle{\problemname}
\begin{block}{题解}
    那么对于本题,我们假设当前的处理右端点为$i$的询问,而线段树上查询$[j,i]$的区间和表示以$j$为左端点的区间的答案

    从$i-1$到$i$时,需要修改线段树上一部分的位置,我们来考虑最右边新加一个点会产生什么影响

    设$g[i]$表示以$i$为右端点的最长连续上升子串的长度,显然 
    
    $$
    g[i] = \begin{cases}g[i-1]+1 & a[i]=a[i-1]+1\\ 1 & \text{otherwise}\end{cases}
    $$

    那么首先$[i-g[i]+1,i]$这一段区间进行区间$+1$。因为这些以$i$为右端点的连续上升的子串都是因为这个新的点而出现的
\end{block}
\end{frame}

\begin{frame}\frametitle{\problemname}
    \begin{block}{题解}
    但是题面要求本质不同,这些串可能在前面出现过,需要把他们在线段树上“删掉”,所以需要一些栈去维护这些旧串的位置

    具体来说就是对于每一种右端点的值$c$开一个栈(可以用链表实现),栈上的每一个元素有两个信息,分别为旧串出现位置以及它的长度区间(其实是三个元素,不过长度区间的上/下限可以通过前一个出栈的元素的长度上/下限推断出来)。
    
    现在我们需要加入一个新的元素入栈,他的位置为$i$,长度为$1 \sim g[i]$。如果当前栈顶元素的长度比$g[i]$小,那就直接出栈且完全在线段树的对应位置上删除,并且重复这个过程。如果比$g[i]$大,那么就不用出栈,同时在线段树上删除被重复的部分。最后再将这个新的元素压入栈顶

    维护完毕后在线段树上查询需要的区间即可
\end{block}
\end{frame}
\renewcommand{\problemname}{J. 传闻档案}

\begin{frame}\frametitle{\problemname}

    \begin{block}{题意}
        给一个带权有向图,每个点价值为此点能够到达的点的最大权值,求所有点价值和。
    \end{block}
\end{frame}

\begin{frame}\frametitle{\problemname}
	\begin{block}{解法}
        反向建边,按点权从大到小进行 dfs 即可。每个点的价值即为这次 dfs 根节点的权值。
        
        注意每个点只访问一次。

        也可以使用 tarjan 缩点,每个连通块的权值变为块中点的最大权值,缩点后 dfs 求出答案。
    \end{block}
\end{frame}
\iffalse
\fi

\end{document}